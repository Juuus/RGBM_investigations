\documentclass[11pt]{article}
%\documentclass[12pt]{article}
%\documentclass[12pt]{article}
%\documentclass[12pt,a4paper]{article}

\usepackage[percent]{overpic}
\usepackage{float}
\usepackage{pgfplots}
%\usepackage[cmbold]{mathtime}
%\usepackage{mt11p}
\usepackage{placeins}
\usepackage{amsmath}
\usepackage{amsthm}
\usepackage{color}
\usepackage{amssymb}
\usepackage{mathtools}
\usepackage{subfigure}
\usepackage{multirow}
\usepackage{epsfig}
\usepackage{listings}
\usepackage{enumitem}
\usepackage{rotating,tabularx}
%\usepackage[graphicx]{realboxes}
\usepackage{graphicx}
\usepackage{graphics}
\usepackage{epstopdf}
\usepackage{longtable}
\usepackage[pdftex]{hyperref}
%\usepackage{breakurl}
\usepackage{epigraph}
\usepackage{xspace}
\usepackage{amsfonts}
\usepackage{eurosym}
\usepackage{ulem}
\usepackage{footmisc}
\usepackage{comment}
\usepackage{setspace}
\usepackage{geometry}
\usepackage{caption}
\usepackage{pdflscape}
\usepackage{array}
\usepackage[round]{natbib}
\usepackage{booktabs}
\usepackage{dcolumn}
\usepackage{mathrsfs}
%\usepackage[justification=centering]{caption}
%\captionsetup[table]{format=plain,labelformat=simple,labelsep=period,singlelinecheck=true}%

%\bibliographystyle{unsrtnat}
\bibliographystyle{aea}
\usepackage{enumitem}
\usepackage{tikz}
\usetikzlibrary{decorations.pathreplacing}
%\def\checkmark{\tikz\fill[scale=0.4](0,.35) -- (.25,0) -- (1,.7) -- (.25,.15) -- cycle;}
%\usepackage{tikz}
%\usetikzlibrary{snakes}
%\usetikzlibrary{patterns}

%\draftSpacing{1.5}

\usepackage{xcolor}
\hypersetup{
colorlinks,
linkcolor={blue!50!black},
citecolor={blue!50!black},
urlcolor={blue!50!black}}

%\renewcommand{\familydefault}{\sfdefault}
%\usepackage{helvet}
%\setlength{\parindent}{0.4cm}
%\setlength{\parindent}{2em}
%\setlength{\parskip}{1em}

%\normalem

%\doublespacing
\onehalfspacing
%\singlespacing
%\linespread{1.5}

\newtheorem{theorem}{Theorem}
\newcommand{\bc}{\begin{center}}
\newcommand{\ec}{\end{center}}
\newtheorem{corollary}[theorem]{Corollary}
\newtheorem{proposition}{Proposition}
\newtheorem{definition}{Definition}
\newtheorem{axiom}{Axiom}
\newcommand{\ra}[1]{\renewcommand{\arraystretch}{#1}}

\newcommand{\E}{\mathrm{E}}
\newcommand{\Var}{\mathrm{Var}}
\newcommand{\Corr}{\mathrm{Corr}}
\newcommand{\Cov}{\mathrm{Cov}}

\newcolumntype{d}[1]{D{.}{.}{#1}} % "decimal" column type
\renewcommand{\ast}{{}^{\textstyle *}} % for raised "asterisks"

\newtheorem{hyp}{Hypothesis}
\newtheorem{subhyp}{Hypothesis}[hyp]
\renewcommand{\thesubhyp}{\thehyp\alph{subhyp}}

\newcommand{\red}[1]{{\color{red} #1}}
\newcommand{\blue}[1]{{\color{blue} #1}}

%\newcommand*{\qed}{\hfill\ensuremath{\blacksquare}}%

\newcolumntype{L}[1]{>{\raggedright\let\newline\\arraybackslash\hspace{0pt}}m{#1}}
\newcolumntype{C}[1]{>{\centering\let\newline\\arraybackslash\hspace{0pt}}m{#1}}
\newcolumntype{R}[1]{>{\raggedleft\let\newline\\arraybackslash\hspace{0pt}}m{#1}}

%\geometry{left=1.25in,right=1.25in,top=1.25in,bottom=1.25in}
\geometry{left=1in,right=1in,top=1in,bottom=1in}

\epstopdfsetup{outdir=./}

\newcommand{\elabel}[1]{\label{eq:#1}}
\newcommand{\eref}[1]{Eq.~(\ref{eq:#1})}
\newcommand{\ceref}[2]{(\ref{eq:#1}#2)}
\newcommand{\Eref}[1]{Equation~(\ref{eq:#1})}
\newcommand{\erefs}[2]{Eqs.~(\ref{eq:#1}--\ref{eq:#2})}

\newcommand{\Sref}[1]{Section~\ref{sec:#1}}
\newcommand{\sref}[1]{Sec.~\ref{sec:#1}}

\newcommand{\Pref}[1]{Proposition~\ref{prop:#1}}
\newcommand{\pref}[1]{Prop.~\ref{prop:#1}}
\newcommand{\preflong}[1]{proposition~\ref{prop:#1}}

\newcommand{\Aref}[1]{Axiom~\ref{ax:#1}}
\newcommand{\Dref}[1]{Definition~\ref{def:#1}}

\newcommand{\clabel}[1]{\label{coro:#1}}
\newcommand{\Cref}[1]{Corollary~\ref{coro:#1}}
\newcommand{\cref}[1]{Cor.~\ref{coro:#1}}
\newcommand{\creflong}[1]{corollary~\ref{coro:#1}}

\newcommand{\etal}{{\it et~al.}\xspace}
\newcommand{\ie}{{\it i.e.}\xspace}
\newcommand{\eg}{{\it e.g.}\xspace}
\newcommand{\etc}{{\it etc.}\xspace}
\newcommand{\cf}{{\it c.f.}\xspace}
\newcommand{\ave}[1]{\left\langle#1 \right\rangle}
\newcommand{\person}[1]{{\it \sc #1}}

\newcommand{\AAA}[1]{\red{{\it AA: #1 AA}}}
\newcommand{\YB}[1]{\blue{{\it YB: #1 YB}}}

\newcommand{\flabel}[1]{\label{fig:#1}}
\newcommand{\fref}[1]{Fig.~\ref{fig:#1}}
\newcommand{\Fref}[1]{Figure~\ref{fig:#1}}

\newcommand{\tlabel}[1]{\label{tab:#1}}
\newcommand{\tref}[1]{Tab.~\ref{tab:#1}}
\newcommand{\Tref}[1]{Table~\ref{tab:#1}}

\newcommand{\be}{\begin{equation}}
\newcommand{\ee}{\end{equation}}
\newcommand{\bea}{\begin{eqnarray}}
\newcommand{\eea}{\end{eqnarray}}

\newcommand{\bi}{\begin{itemize}}
\newcommand{\ei}{\end{itemize}}

\newcommand{\Dt}{\Delta t}
\newcommand{\Dx}{\Delta x}
\newcommand{\Epsilon}{\mathcal{E}}
\newcommand{\etau}{\tau^\text{eqm}}
\newcommand{\wtau}{\widetilde{\tau}}
\newcommand{\xN}{\ave{x}_N}
\newcommand{\Sdata}{S^{\text{data}}}
\newcommand{\Smodel}{S^{\text{model}}}

\newcommand{\del}{D}
\newcommand{\hor}{H}
\newcommand{\subhead}[1]{\mbox{}\newline\textbf{#1}\newline}

\setlength{\parindent}{0.0cm}
\setlength{\parskip}{0.4em}

\numberwithin{equation}{section}
\DeclareMathOperator\erf{erf}
%\let\endtitlepage\relax

\begin{document}

%\onehalfspacing
\begin{titlepage}
\title{Mobility, Mixing and Ergodicity: A Physically-Motivated Measure for Economic Mobility}
\author{Viktor Stojkoski \footnote{Macedonian Academy of Sciences and Arts,~\url{vstojkoski@manu.edu.mk}} \and Alexander Adamou\footnote{London Mathematical Laboratory,~\url{a.adamou@lml.org.uk}} \and Yonatan Berman\footnote{London Mathematical Laboratory,~\url{y.berman@lml.org.uk}} \and Colm Connaughton \footnote{London Mathematical Laboratory and University of Warwick,~\url{c.p.connaughton@warwick.ac.uk}} \and Ole Peters\footnote{London Mathematical Laboratory and Santa Fe Institute,~\url{o.peters@lml.org.uk}}\,\, \thanks{We thank...}}
%\date{First version: August 26, 2018\,\,\,\,\,\,\,\,\,\,\,\,\,\,\,\,\,\,\,\,\,\,\,\,Last revised: \today}
%\date{}
\date{\today}
\maketitle

%\bc
%\red{Preliminary version, please do not circulate}
%\ec

\begin{abstract}
\noindent 
\\
\\
\noindent\textbf{Keywords: mobility, inequality, ergodicity economics}
%\\

%\bigskip
\end{abstract}
\setcounter{page}{0}
\thispagestyle{empty}
%\nopagebreak
\end{titlepage}
\pagebreak \newpage
%\nopagebreak

\section{Introduction}\label{sec:introduction}

% What is mobility?
Economic mobility describes ``dynamic aspects of inequality.''~\citep{Shorrocks1978} In the intergenerational context it seeks ``to measure the degree to which a child's social and economic opportunities depend on his parents' income or social status.''~\citep{chettyETAL2014} Mobility measures traditionally quantify how wealth (or income\footnote{We focus on wealth in this paper, but it applies also to income}) ranks of individuals evolve over time. Intuitively, when mobility is high, ranks evolve quickly, and the chances of an individual to change her position in the wealth distribution over a given time period are high. When mobility is low, individuals are unlikely to change their rank in the distribution over time, or that it changes slowly.

% How is mobility typically measured?
\citet{Shorrocks1978} described several desired properties of statistical measures of mobility and set the standard for such measures. Mobility measures are assumed to be derived from a transition matrix, a bistochastic matrix describing the conditional probabilities of individuals to move between different ranks over some time period. An example for such a measure, used extensively in the mobility literature, is the rank correlation, the correlation between individual wealth ranks in two points in time. Another canonical measure of mobility, used most typically in studies of intergenerational income mobility, is the intergenerational earnings elasticity (IGE), defined as the regression coefficient between log-incomes of parents and children.\footnote{In fact, the rank correlation and the IGE are both measures of immobility, and to consider them as measures of mobility one has to subtract them from 1.}

% Why are the typical measures problematic?
The standard statistical measures of mobility have several limitations. First, they are generally incomparable. For example, a rank correlation of 0.3 would have a different meaning if it corresponds to a period of one year or of ten years. Second, for a given time period, it is not generally possible to tell whether the rank correlation is high or low, as it is a dimensionless quantity between -1 and 1. The rank correlation over some time period can only be high or low in comparison to other economies over a similar time period, or to other time periods of the same length.

The interpretation of the rank correlation also depends on the underlying wealth distribution and its dynamics. For example, the same rank correlation cannot be interpreted similarly when the underlying wealth distribution remains unchanged over time, and when it becomes less equal over time.

The IGE has similar limitations, but also others. Most notably, the IGE is sensitive by-design to the level of inequality, \ie to the shape of the underlying distributions, and not only to the transition matrix, as discussed in detail in the mobility literature (\eg~\citet{chettyETAL2014}).

% This paper
This paper introduces mixing time, a property of stochastic processes, as a measure of mobility. When wealth is an ergodic observable~\citep{PetersAdamou2018c}, and assuming that the wealth distribution approaches a steady state, if the wealths of an arbitrary subgroup of individuals are followed over time, the distribution of wealth within this subgroup will converge to the steady-state wealth distribution over time. The convergence time of this process is the mixing time. Put simply, it is the time scale over which individuals mix into the wealth distribution.

% Mixing times solve the comparability issue
Using mixing time as a measure for mobility overcomes the comparability issue. If we are interested in mobility over a specific time window -- one year, ten years, one generation, \etc -- determining whether mobility is high or low is immediate by comparing the time window to the observed mixing time. When mixing is rapid, \ie the mixing time is short relative to the window of observation, we interpret mobility as high. Slow mixing is interpreted as low mobility.

% We study RGBM
We then consider Reallocating Geometric Brownian Motion (RGBM~\citep{MarsiliMaslovZhang1998,LiuSerota2017,BermanPetersAdamou2019}) as a model for wealth dynamics and study mixing in this model. In RGBM, individual wealth undergoes random multiplicative growth, modeled as Geometric Brownian Motion (GBM), and is reallocated among individuals by a simple pooling and sharing mechanism. RGBM is a null model of an exponentially growing economy with social structure. It has three parameters representing economic growth, random shocks to individual wealth, and economic interaction among agents, quantified by a reallocation rate. This model is known to reproduce several important stylized facts. In particular, when the reallocation rate is positive, the wealth distribution converges to a stationary distribution with a Pareto tail. The model has both ergodic and non-ergodic regimes, characterized by the sign of the reallocation rate parameter~\citep{BermanPetersAdamou2019}.

% What is the mixing time in RGBM?
We find that in RGBM the mixing time scales with the inverse of the reallocation rate. As the reallocation rate increases, \ie when a larger share of each individual's wealth is pooled and shared per unit time, the mixing time becomes proportionally shorter, and mobility increases. As the reallocation rate approaches zero, the mixing times get longer, and mobility gets lower. Since decreasing reallocation rates also lead to increasing inequality, this result is in line with the empirical observation that as inequality increases mobility decreases, and vice versa~\citep{corak2013}.

% Mixing time and standard measures
\YB{Here we need to describe how mixing time and rank correlation or other standard measures are related in RGBM}

% Mixing in non-ergodic systems
In practice, many economic observables are best modeled as non-ergodic~\citep{Peters2019b}. In particular,~\citet{BermanPetersAdamou2019} argue that the US economy is best described in RGBM as one in which wealth is systematically reallocated from poorer to richer, \ie the reallocation rate is negative. In such a case there is no mixing, so the mixing time is infinite. Thus, measuring mobility using standard measures under this regime is misleading. The thorough study of RGBM in this regime is outside of the scope of this paper and left for future work.  

% Plan
The paper is organized as follows.~\Sref{mixingtime} discusses the concept of mixing time and how it provides a physically-motivated measure for mobility.~\Sref{rgbm} studies mobility using mixing times in reallocating geometric Brownian motion as a model for wealth. We conclude in~\Sref{conclusion}.

\section{Mixing time as a measure of mobility}\label{sec:mixingtime}

\subsection{What is a mixing time?}\label{sec:what}

\section{Mixing in reallocating geometric Brownian motion}\label{sec:rgbm}

\subsection{Mixing time and standard measures of mobility}\label{sec:measures}

\subsection{Mobility and inequality in reallocation geometric Brownian motion}\label{sec:inequality}

\section{Conclusion}\label{sec:conclusion}



\bibliography{../LML_bibliography/bibliography}

%\clearpage
%\appendix

%\section{}\label{app:appA}

\end{document}